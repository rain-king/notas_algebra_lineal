\documentclass{article}
\usepackage{graphicx} % Required for inserting images
\usepackage{amsmath}
\usepackage{hyperref}
\usepackage[utf8]{inputenc}

\counterwithin*{equation}{section}
\counterwithin*{equation}{subsection}

\title{Notas de Algebra Lineal}
\author{Murillo Vega, Gustavo \\ e-mail:
\href{mailto:g.murillo24@info.uas.edu.mx}{g.murillo24@info.uas.edu.mx}}
\date{20 de Octubre 2024}

\begin{document}

\maketitle

\section{Introducción}
\subsection{Definición de matriz}
Sea ${F}$ el campo de los reales o de los complejos, definimos una matriz $A\in {F}^{m\times n}$ por los puntos 1 a 4.
\begin{enumerate}
    \item Denotamos sus elementos por $A_{i,j}\in A$ con  $i=1,\ldots,m$ y $j=1,\ldots,n$. También, podemos denotar una matriz por sus elementos:
    
    $(a_{i,j})_{m\times n} \in {F}^{m\times n}$, donde redundantemente $((a_{i,j})_{m\times n})_{i,j} := a_{i,j}$.\\
    Cuando no haya confusión con las dimensiones de la matriz
    escribimos\\ $(a_{i,j}) := (a_{i,j})_{m\times n}$.
    
    \item Definimos la suma de matrices $A+B = C$ para
    \\ $A, B \in {F}^{m\times n}$ como:
    $$C_{i,j}=A_{i,j} + B_{i,j}.$$

    \item Se define el producto $cA$ de $A$ por el escalar $c$
    como $$(cA)_{i,j} = cA_{i,j}.$$
    
    \item Como se verá en seguida, es útil definir el producto
    $AB\in {F}^{m\times n}$ de las matrices $A\in {F}^{m\times d}$ y
    $B \in {F}^{d\times n}$ de la siguiente manera:
    $$(AB)_{i,j} = \sum^d_{k=1} A_{i,k} B_{k,j}.$$
    
    Nota: Si las dimensiones de $A$ y $B$ son $m\times d$
    y $d\times n$, el producto $AB$ tiene dimensiones $m\times n$

    \item Si $A\in F^{m\times n}$, definimos la transpuesta $A^t$
    de $A$ por ${A^t}_{i,j} = A_{j,i}$.
    Y la adjunta de $A$ como $A^* = \overline{A^t}$.

    \item \textit{Notación de arreglo:} Si
    $A\in {F}^{m\times n}$, escribimos
    $$A = \begin{bmatrix}
        A_{1,1} & A_{1,2} & \ldots & A_{1, n} \\
        A_{2,1} & A_{2,2} & \ldots & A_{2, n} \\
        \vdots & \vdots & \vdots & \vdots \\
        A_{m,1} & A_{m,2} & \ldots & A_{m, n}
    \end{bmatrix}.$$
\end{enumerate}

\subsection{Identificación de vectores columna y escalares}
Un una matriz $x$ de dimensión $m\times 1$, en el contexto de matrices, se dice que $x$ es un vector $m$-dimensional en $F^{m\times 1}$, y denotamos $x_i := x_{i,j}$
pues $j$ siempre es $1$. También identificamos
$F^{m\times 1}$ con ${F}^m$. Por lo tanto,
al hablar de vectores en el contexto de matrices, un vector
es una matriz que consiste de una sola columna.

Sea $\alpha\in F^{1\times 1}$, identificamos $\alpha_{i,j}$ con
$\alpha$ y $F$ con $F^{1\times 1}$, es decir identificamos $\alpha$
como un escalar.

\subsection{Ecuaciones lineales en forma de matriz}
El sistema de ecuaciones lineales general es:
\begin{align*}
    \sum^n_{j=1}a_{1,j} x_j &= b_1 \\
    \sum^n_{j=1}a_{2,j} x_j &= b_2 \\
    &\vdots \\
    \sum^n_{j=1}a_{m,j} x_j &= b_m.
\end{align*}

Ahora, veamos la forma de la multiplicación de
$A=(a_{i,j})\in F^{m\times n}$
por $x=(x_1, x_2, \ldots, x_n)^t \in F^{n\times 1}$:
$$Ax = 
\begin{bmatrix}
    a_{1,1} & a_{1,2} & \ldots & a_{1, n} \\
    a_{2,1} & a_{2,2} & \ldots & a_{2, n} \\
    \vdots & \vdots & \vdots & \vdots \\
    a_{m,1} & a_{m,2} & \ldots & a_{m, n}
\end{bmatrix}
\begin{bmatrix}
    x_1 \\
    x_2 \\
    \vdots \\
    x_n
\end{bmatrix}
=
\begin{bmatrix}
    \sum^n_{j=1} a_{1,j} x_j \\ 
    \sum^n_{j=1} a_{2,j} x_j \\
    \vdots \\
    \sum^n_{j=1} a_{m,j} x_j \\ 
\end{bmatrix}.$$
Si inspeccionamos esta expresión con la del sistema de ecuaciones lineales, tenemos con $b=(b_1, b_2, \ldots, b_m)^t$, que el mismo sistema de ecuaciones se puede expresar como
$$Ax = b.$$

\subsection{Algunos productos usados más adelante}
\textbf{Producto interno}

Si $x$ y $y$ son vectores de la misma dimensión, definimos su
producto interior $(x,y)$ como
\[(x,y) = x^t \overline y\]
o bien
\[(x,y) = y^* x\]
\\
\textbf{Producto exterior}

Si $x=(x_1,\ldots,x_n)^t$ y $y=(y_1,\ldots, y_n)^t$, entonces
su producto exterior es
$$x y* =
\begin{bmatrix} x_1 \\ \vdots \\ x_n \end{bmatrix}
\begin{bmatrix}
    \overline{y_1} & \ldots & \overline{y_n}
\end{bmatrix} =
\begin{bmatrix}
    x_1 \overline {y_1} & \ldots & x_1 \overline {y_n} \\
    x_2 \overline{y_2} & \ldots & x_2 \overline{y_n} \\
    \vdots & \vdots & \vdots \\
    x_n \overline{y_1} & \ldots & x_n \overline{y_n}
\end{bmatrix}.
$$
\newpage

\section{Espacios Vectoriales}
(Usamos la palabra mapeo y transformación de forma intercambiable.)

Ya que vimos que los sistemas de ecuaciones lineales se pueden expresar con matrices, es de interés saber qué tipo de matrices generan sistemas de ecuaciones con soluciones, solución única o ninguna solución, para esto necesitamos la herramienta de mapeos lineales, que serán análogos a la función que corresponde a la multiplicación por la izquierda de una matriz con un vector columna. Sin embargo, estas son funciones de un espacio vectorial a otro, por lo que primero tenemos que definir qué es un espacio vectorial.

Como nota adicional, diremos que puede ser largamente innecesario para los practicantes de ciencias e ingeniería considerar mapeos sobre espacios vectoriales abstractos, en lugar de las transformaciones dadas por matrices sobre los vectores columna. Esto porque como se puede demostrar, tales ambas transformaciones son equivalentes por medio de isomorfismos.

Sin embargo para desarrollar teoría del álgebra lineal, e incluso modelos dentro de la ciencia y la ingeniería, es prolifero considerar estas transformaciones y espacios vectoriales abstractamente.

\subsection{Definición de espacio vectorial}
Un espacio vectorial $V(F)$ sobre un campo $F$, también denotado $V$ cuando el campo es implicado, tiene elementos que cumplen las siguientes propiedades:
\begin{enumerate}
    \item \textbf{Suma y multiplicación escalar}
    
    Si $x,y\in V(F)$ y $\alpha \in F$, existe $x+\alpha y \in V$.
    Donde el operador $+$ es conmutativo y asociativo.
    
    \item \textbf{Existencia del 0}
    
    Existe $0\in V$ tal que si $x\in V$, $x+0 = x$.

    \item \textbf{Identidad escalar}
    
    $1x = x$ para $x\in V$, por virtud de $0+1x$.

    \item \textbf{Existencia del inverso aditivo}

    Si $x\in V$ existe $-x\in V$ tal que $x+(-x) = 0$.
    
    \item \textbf{Asociatividad de vector con escalar}

    Si $\alpha, \beta\in F$ y $x\in V$,
    $(\alpha\beta)x = \alpha (\beta x)$.
    
    \item \textbf{Distributividad}

    Si $\alpha,\beta\in F$ y $x,y\in V$
    \begin{itemize}
        \item $(\alpha+\beta)x = \alpha x + \beta x$.
        \item $\alpha (x + y) = \alpha x + \alpha y$.
    \end{itemize}
\end{enumerate}

Los elementos de un espacio vectorial se llaman vectores.

\subsubsection*{Ejemplos}
Esto implica que el conjunto de matrices $F^{m\times n}$ con sus
operaciones (excluyendo la multiplicación) son un espacio vectorial $V(F)$.

En particular $F^{m\times 1}$ es un espacio vectorial, de donde
en ingeniería o ciencias a estos elementos son a los que generalmente se llaman vectores. Notemos la diferencia, que aunque estos son vectores en el sentido del espacio vectorial también son matrices columna (o vectores en el sentido matricial). Pero también, las matrices son vectores en su propio espacio vectorial.

El conjunto de funciones $f: X \rightarrow F$ de un conjunto $X$
a un campo $F$ forman un campo vectorial. En particular, los polinomios $P^n(t)$ sobre un campo $F$ y $t$ en un anillo $R$ con identidad (ejemplo, las matrices con todas sus operaciones, o algún campo) son un espacio vectorial. Para aclarar, un polinomio de tal espacio tiene
la forma $\sum^n_{i=1} \alpha_i t^n$ con $\alpha_i\in F$.

\subsection{Subespacio vectorial}
Un subespacio vectorial $U$ de $V$ es un espacio vectorial
con el mismo campo y operaciones que $V$, tal que $U\subset V$.

\subsection{Suma de subespacios}
Sean $V_j$ $m$ subespacios de $V$, entonces definimos el conjunto
de $\sum_j v_j$ con $v_j\in V_j$ como la suma de los $V_j$ subespacios
de $V$, y lo denotamos $\sum_j V_j$.
\subsubsection*{Propiedades}
\begin{enumerate}
    \item $\sum_j V_j$ es un subespacio de $V$
    \item $\sum_j V_j\subset U$ si $V_j\subset U$ para cada $j$.
    Es decir, la suma de subespacios, es el subespacio más pequeño
    que contiene a todos los subespacios de la suma.
\end{enumerate}

\subsection{Suma directa}
$\sum_j V_j$ es una suma directa si $x\in\sum_j V_j$
si solo hay una forma de escribir $x$ como  $\sum_j v_j$
con $v_j\in V_j$. Entonces escribimos $\oplus_j V_j$
o $$V_1\oplus\cdots\oplus V_n.$$

\subsubsection*{Propiedades}
\begin{enumerate}
    \item $\sum_j V_j$ es una suma directa si y sólo si
    la única forma de escribir $0$ como $\sum_j v_j$
    con $v_j\in V_j$ es siendo $v_j=0$ para cada $j$.
    \item Sean $U$ y $W$ subespacios de $V$, entonces
    $U+W$ es suma directa si y sólo si $U\cap W = \{0\}$.
\end{enumerate}

\newpage

\section{Bases y dimensiones}
\subsection{Combinación lineal}
Una combinación lineal de vectores $x_i \in V$ es el vector $\sum_i \alpha_i x_i$ con $\alpha_i\in F$.

\newcommand{\spn}{\text{span }}
\subsection{Conjunto generador y espacio generado}
Sea $U\subset V$, se dice que $U$ es el conjunto generador de $W$,
o equivalentemente que $W$ es generado por $U$, si 
$W$ es el conjunto de todas las combinaciones lineales de $U$.
Denotamos tal $W$ por $\spn U$.

Nótese que $\spn U$ es un subespacio de $V$.

\subsection{Independencia lineal}
Un conjunto de vectores $x_i$ es independiente linealmente si
cuando una combinación lineal de ellos es $0$, todos los coeficientes
correspondientes $\alpha_i$ son igual a $0$. Es decir
$$\sum_i \alpha_i x_i \implies \alpha_i = 0.$$

\subsubsection*{Ejemplo}
El conjunto que consiste de $0$ no es linealmente independiente, pues
$\alpha 0 = 0$ no implica que $\alpha = 0$.

El conjunto de un elemento distinto de $0$, $v$, es linealmente independiente.

\newpage
\subsubsection{Equivalencia de independencia lineal}\label{dependencia}
Un vector $x\in V$ es combinación lineal de los vectores
$x_1,\ldots,x_n$, si y sólo si $x_1,\ldots,x_n,x$ no son linealmente
independientes.
\subsubsection*{Prueba}
Si $x=\sum_i \alpha_i x_i$ (es combinación lineal de los $x_i$), tenemos
$$\sum_i \alpha_i x_i +(-1)x = 0$$
pero esta combinación lineal igual a $0$ tiene un coeficiente que
nunca es $0$\\
(i.e $-1$), luego, $x_1,\ldots,x_n,x$ no son linealmente
independientes.

Ahora supongamos que $x_1,\ldots,x_n,x$ son linealmente independientes.
$$\sum_i \alpha_i x_i + \alpha x = 0$$
implica que $\alpha_i$ y $\alpha$ son $0$. De manera que
$$\sum_i \alpha_i x_i = x$$
o
$$\sum_i \alpha_i x_i + (-1)x$$
implicaría que $-1=0$. Por lo tanto $x$ no es combinación lineal
de los $x_i$. \hfill ////

\subsubsection{Eliminar un vector de una lista linealmente dependiente}
\label{eliminación}
Si $x_1,\ldots,x_n$ es una lista de vectores linealmente dependientes que genera
a $V$, eliminar uno de ellos sigue generando $V$
\subsubsection*{Prueba}
Por hipótesis todo $x\in V$ puede escribirse como
$$\sum^n_{i=1} \alpha_i x_i.$$
Por el teorema anterior, $x_n$ puede escribirse como
combinación lineal de los $x_1,\ldots,x_{n-1}$ restantes,
sea
$$x_n=\sum^{n-1}_{j=1}\beta_j x_j$$
luego
\begin{align*}
    x&=\sum^{n-1}_{i=1} \alpha_i x_i +
    \alpha_n\sum^{n-1}_{j=1}\beta_j x_j\\
    &=\sum^{n-1}_{i=1}(\alpha_i+\alpha_n\beta_i)x_i.
\end{align*}
De manera que $x_1,\ldots,x_{n-1}$ siguen generando $V$.
\hfill ////

\subsection{Base}
Se dice que $B\subset V$ es una base para $V$, si
$\spn B = V$ y los vectores en $B$ son linealmente
independientes.

\subsubsection*{Ejemplo}
El conjunto $O=\{0\}$ genera al espacio vectorial $O(F)$
que consiste de $0$. Sin embargo $O$, como ya vimos, no es linealmente independiente,
por lo tanto no es base para $O(F)$.
Excluimos el caso en que $F$ consiste de $0$, pues entonces $O(F)$ sí es linealmente independiente.
Por supuesto, tal campo no lo tenemos en consideración. Comúnmente por $F$ nos referimos
a los complejos o los reales, y en raras ocasiones (no en estas notas)
a campos finitos o contables.

\subsection{Dimensión}
La dimensión de $V$ es la cardinalidad de una base de $V$.
Esto está bien definido pues se puede probar que todas las bases de
$V$ tienen la misma cardinalidad. 

En el caso de que la base sea
finita, la dimensión es el número de vectores en la base. Pero
también hay espacios vectoriales con bases numerables, 
donde las combinaciones lineales en general son una serie.

Como ya hemos dicho antes, el conjunto que $O$ consiste de $0$ no es base, sin embargo $O$ se genera a sí mismo. Por convención decimos que $O$ tiene dimensión $0$, ya que no tiene una base.
\newpage

\subsection{Propiedades importantes de bases y dimensiones}
\label{bases}
Sea $V$ un espacio vectorial de dimensión finita distinto de $O$.
\begin{enumerate}
    \item Una lista de vectores $v_j$ que genera a $V$ puede
    reducirse a una\\
    base para $V$.
    \item Una lista de vectores $v_j$ linealmente independientes 
    puede extenderse a una base de $V$.
    \item Una lista de $\dim V$ vectores linealmente independientes es una\\
    base para $V$.
\end{enumerate}
\subsubsection*{Prueba}
1. Sea $v_1,\ldots,v_n$ una lista que genere a $V$.
Supongamos que primero, reducimos la lista de $v_i$ a $m\leq n$ 
vectores eliminando los ceros. Esta lista sigue generando a $V$.

Luego, repetimos el siguiente procedimiento $n$ veces:
Si $v_k$ es el último vector de la lista reducida, y es
combinación lineal de $v_1,\ldots,v_{k-1}$, lo eliminamos y repetimos el proceso con $k\leftarrow k-1$. De lo contrario, dejamos la lista
sin modificar. En el primer caso, la lista sigue generando $V$ por
\ref{eliminación}, y obviamente, también en el segundo caso.

Ya que así recorremos toda la lista, ningún vector es combinación
lineal del resto, y por \ref{dependencia} la lista final es 
linealmente independiente, y sigue generando $V$ por el diseño
del procedimiento.

Por lo tanto la lista resultante es una base para $V$. 

2. Sean $v_1,\ldots,v_n$ linealmente independientes,
y $w_1,\ldots, w_m$ generadores de $V$.
Luego $v_1,\ldots,v_n,w_1,\ldots,v_m$ genera a $V$, y por
el punto 1 se puede reducir a una base. En el procedimiento de la
prueba de 1, no se eliminan vectores que no sean combinación lineal del
resto, por lo tanto ningún $v_j$ se elimina. Sigue que la base
tiene todos los $v_j$.

% 3. Sea $n=\dim V$, y $x_1,\ldots,x_n$ linealmente independientes, y sea $y_1,\ldots,y_n$ una base para $V$ (que podemos escribir así,
% pues por definición de dimensión una base tiene $\dim V$ elementos).

% $x_j$ se puede escribir como
% $$\sum_i\alpha_{i,j} y_i$$
% luego cualquier combinación lineal de los $x_j$ es
% \begin{align*}
%     \sum_j\beta_j\sum_i\alpha_{i,j}y_i &= \sum_i
%     \left(\sum_j\beta_j\alpha_{i,j}\right)y_i\\
%     &= \sum_i \lambda_i y_i.
% \end{align*}
% Al ser $\lambda_j$ arbitrario, $x_j$ genera a $V$ y por lo tanto
% es una base para $V$.

3. Sean $x_1,\ldots,x_{\dim V}$ vectores linealmente independientes.
Por el punto 2, estos se pueden extender a una base de $V$,
$x_1,\ldots,x_m$ con $m\geq \dim V$. Sin embargo una base de $V$ tiene
$\dim V$ vectores, por lo tanto $m=\dim V$. Esto quiere decir
que la lista de vectores original es una base para $V$.
\hfill ////
    
\newpage

\section{Mapeos lineales}
\subsection{Motivación}
Nótese que la función $f: F^n \rightarrow F^m$ dada por
$f(x) = Ax$ donde $A\in F^{m\times n}$, tiene las propiedades
\begin{enumerate}
    \item \textbf{Aditividad}

    $f(x+y) = f(x) + f(y)$ para $x,y\in F^n$.

    \item \textbf{Homogenidad}

    $f(\alpha x) = \alpha f(x)$ para $\alpha\in F$.
\end{enumerate}
Generalizamos estas funciones de la siguiente manera.

\subsection{Mapeo lineal}
Una función $T: V\rightarrow W$ con $V,W$ de espacios vectoriales es
un mapeo lineal si cumple con las dos condiciones anteriores, reemplazando $V$ por $F^n$ y $W$ por $F^m$. Usualmente,
denotamos $T(x)$ por $Tx$ para $x\in V$.

\subsection{Espacio de mapeos lineales}
Denotamos por $\mathcal{L}(V,W)$ al conjunto de todas
las transformaciones lineales de $V$ a $W$.
Y denotamos $\mathcal{L}(V)$ a $\mathcal{L}(V, V)$. A estos últimos
les llamamos \textit{operadores lineales}. En particular el operador
$I\in\mathcal{L}(V)$ definido por $Ix=x$ se llama la identidad en $V$.

Se puede demostrar fácilmente que $\mathcal{L}(V,W)$ es un espacio vectorial.

%\newcommand{\ker}{\text{ker }}
\subsection{Kernel}
Decimos que el espacio nulo, o kernel, de $T: V \rightarrow W$, es el conjunto de $x\in V$ tales que $Tx=0$. Y lo denotamos por
$\ker T$.

Podemos probar que $\ker T$ es un subespacio de $V$:
Sea $x \in \spn \ker T$, es decir $x = \sum_i \alpha_i x_i$ donde
$x_i\in \ker T$, entonces $Tx = \sum_i \alpha_i Tx_i = 0$ pues
cada $Tx_i=0$, sigue que $x\in\ker T$.

\subsubsection*{Inyectividad}
En general, una función $f$ es inyectiva si $x\neq y$ implica
$f(x) \neq f(y)$.

En particular, podemos probar que un mapeo lineal $T: V\rightarrow W$
es inyectivo si $\ker T = \{0\}$: Sean $x\neq y$ vectores de $V$,
de manera que $x-y\neq 0$, y supóngase que $\ker T$ consiste de $0$. Luego $T(x-y) \neq 0$, pues de lo contrario implicaría $x-y = 0$,
de manera que $T(x-y) = Tx - Ty \neq 0$ o $Tx \neq Ty$, y $T$ es
inyectiva.

El regreso también es verdadero: Sea $T: V\rightarrow W$ inyectiva,
la contrapuesta de la definición de inyectividad nos dice que
dados $x,y\in V$
$Tx = Ty$ implica $x = y$, esto es $T(x-y) = 0$ implica $x-y = 0$.
Luego $\ker T$ consiste de $0$.

\textbf{Resultado importante:}
El mapeo lineal $T$ es inyectivo ssi $\dim\ker T = 0$.

\newcommand{\im}{\text{im }}
\subsection{Imagen}
Dado $T: V\rightarrow W$, la imagen de $T$ es
$T(V) = \{Tx: x\in V\}$. Y la denotamos por $\im T$.

Se puede demostrar, que análogo a $\ker T$, $\im T$ forma un
subespacio de $W$: Sea $y\in \spn\im T$, entonces
$y=\sum_i \beta_i y_i$ donde $y_i=Tx_i$ para algún $x_i\in V$.
Luego $y=T(\sum_i \beta_i x_i) = Tx$ para
$x=\sum_i \beta_i x_i \in V$; $y\in\im T$.

\subsubsection*{Sobreyectividad}
Trivialmente, $T: V\rightarrow W$ es sobreyectiva si $\im T = W$.

\subsection{Importancia del kernel y la imagen}
Si tenemos el sistema de ecuaciones lineales homogéneo, $Ax=0$,
es de interés saber si hay otras soluciones además de $x=0$.
Esto se puede responder si encontramos que el mapeo $T_A: F^n \rightarrow F^m$
dada por $T_A(x) = Ax$, tiene kernel distinto de $\{0\}$.

Si tenemos un sistema de ecuaciones lineales $Ax=b$, es de
interés si existen soluciones para $x$, y si existe alguna,
si es única. Esto se podrá responder con teoremas a partir de propiedades tanto la imagen como del kernel de la transformación $T_A$.

\subsection{Teorema fundamental de los mapeos lineales}
Sea $T: V\rightarrow W$ un mapeo lineal con $V$ de dimensión finita,
entonces
$$\dim V = \dim\ker T + \dim\im T.$$

\subsubsection*{Prueba}
Formen $u_i$ con $i$ de $1$ hasta $n$ una base para $\ker T$,
al ser subespacio de $V$, esta base se puede extender a una base $B$
de $V$. Donde $B$ consiste de los $n$ $u_i$, y
$v_j$ con $j$ de $1$ hasta $m$. De manera que la dimensión de $V$
es $\dim\ker T+m$.

Ahora consideremos $x\in V$, expandido de la base $B$, es decir\\
$x= \sum^n_{i=1}\alpha_iu_i + \sum^m_{j=1}\beta_jv_j$. Luego
$Tx = \sum^m_{j=1}\beta_jTv_j$. Queremos mostrar que $Tv_j$ de
$1$ hasta $m$ forma una base para $\im T$. Primero, como $x$ era arbitrario, $\spn \{Tv_j\}_j = \im T$.

Sea $\sum^m_{j=1}\beta_jTv_j = T(\sum_j\beta_jv_j)=0$.
Entonces $\sum_j\beta_jv_j \in \ker T$, y escrito respecto de la base
de los $u_i$,
$$\sum_j\beta_jv_j = \sum_i\alpha_iu_i$$
luego
$$\sum_j\beta_jv_j + \sum_i(-\alpha_i)u_i = 0$$
pero como los $v_j$ con los $u_i$ forman una base para $V$,
son linealmente independientes y $\beta_j=\alpha_i=0$.

Hemos probado que $\sum^m_{j=1}\beta_jTv_j = 0$ implica
$\beta_j = 0$, es decir, la independencia lineal de los $Tv_j$.
\hfill ////

\subsubsection{Corolario}
Un mapeo lineal $T:V\rightarrow W$ a un espacio de menor dimensión no es inyectivo.
\subsubsection*{Prueba}
\begin{align*}
\dim\ker T &= \dim V - \dim\im T\\
    &\geq \dim V - \dim W\\
    &> 0.
\end{align*}
Como probamos antes, $T$ es inyectiva solo si $\dim \ker T = 0$,
por lo tanto no es inyectivo.\hfill ////

\subsubsection{Corolario}
Un mapeo lineal $T: V\rightarrow W$ a un espacio de dimensión mayor
no es sobreyectivo.

\subsubsection*{Prueba}
\begin{align*}
    \dim\im T &= \dim V - \dim\ker T\\
    &< \dim W - \dim\ker T \\
    &\leq \dim W.
\end{align*}
Entonces $\im T \neq W$, es decir, $T$ no es sobreyectivo.
\hfill ////

\subsection{Invertibilidad}
Una función invertible es una función inyectiva y sobreyectiva.
Se demuestra en cursos de álgebra superior que esta definición es
equivalente a que para \\
$f:A\rightarrow B$ exista una función
$g:B\rightarrow A$ tal que $(f\circ g)(y) = y$ para $y\in B$,
y $(g\circ f)(x) = x$ para $x\in A$.
\newpage

\subsubsection{Invertibilidad en espacios de la misma dimensión}
Sean $\dim V = \dim W < \infty$.
\begin{enumerate}
    \item Si $T\in\mathcal{L}(V,W)$ entonces
    inyectividad, invertibilidad y sobreyectividad son equivalentes.
    \item Si $T\in\mathcal{L}(V,W)$ y $S\in\mathcal{L}(W,V)$,
    $ST=I$ o $TS=I$, implica la invertibilidad de ambos mapeos.
\end{enumerate}
\subsubsection*{Prueba}
\begin{enumerate}
    \item Si $T$ es inyectiva, $\ker T$ tiene dimensión $0$, de manera 
    que\\
    $\dim V = \dim \im T$, por hipótesis $\dim V = \dim W$ y
    $\dim \im T = \dim W$, es decir, $T$ es sobreyectiva, y por lo tanto
    invertible.
    
    Si $T$ es sobreyectiva, $\im T = W$, de manera que
    $$\dim V = \dim \ker T + \dim W \implies \dim \ker T = 0.$$
    Es decir, $T$ es inyectiva y por lo tanto invertible.

    \item Sea $ST=I$, y $v\in\ker T$, entonces
    $$v = (ST)v = S(Tv) = S0 = 0$$
    por lo que $T$ es inyectiva al ser $\ker T$ de dimensión $0$,
    y por lo tanto invertible.

    Multiplicando $ST$ por $T^{-1}$ por la derecha llegamos a que,
    $S=T^{-1}$, y por ser la inversa de otra función, es invertible. 

    Análogamente si $TS=I$ se sigue el mismo procedimiento con
    $w\in\ker S$, y se llega a que $S$ es invertible. Luego a que $T$ 
    es invertible. \hfill ////
\end{enumerate}

\subsubsection{La inversa de la multiplicación}
Sean $T\in\mathcal{L}(V,W)$ y $S\in\mathcal{L}(W,V)$ invertibles, 
entonces $ST$ es invertible y $(ST)^{-1}=T^{-1}S^{-1}$.
\subsubsection*{Prueba}
Existen $S^{-1}$ y $T^{-1}$, entonces
\begin{align*}
    ST(T^{-1}S^{-1}) &= S(TT^{-1})S^{-1}\\
    &= SIS^{-1}\\
    &= SS^{-1} = I.
\end{align*}
\hfill ////

\subsubsection{Espacios vectoriales isomorfos}
Se dice que un espacio vectorial $V$ es isomorfo a otro $W$,
si existe un mapeo lineal invertible $T$ de $V$ a $W$. Usualmente en 
este contexto, a $T$ se le llama isomorfismo de $V$ a $W$.

\subsubsection{La dimensión indica isomorfidad}
Dos espacios de dimensión finita son isomorfos
si y sólo si son de la misma dimensión.
\subsubsection*{Prueba}
Sea $T\in\mathcal{L}(V,W)$ un isomorfismo, porque es invertible es 
inyectivo y sobreyectivo, es decir $\dim\ker T = 0$ y
$\dim\im T = \dim W$, por el teorema fundamental de mapeos lineales
sigue $$\dim V = \dim W.$$

Ahora si $\dim V = \dim W$, al ser de la misma dimensión, sea $n$, las 
bases de $V$ y $W$ se pueden escribir como $v_i$ y $w_i$ 
respectivamente con $i$ de $1$ a $n$. Entonces existe un 
$T\in\mathcal{L}(V,W)$ definido por
$$T\left(\sum^n_{i=1}c_i v_i\right) = \sum^n_{i=1}c_i w_i$$
con $c_i\in F$. Este $T$ es claramente sobreyectivo pues $Tx$ es una
expansión de la base del contradominio. Entonces es invertible, y
$V$ isomorfo a $W$. \hfill ////

\newpage

\section{Matrices}
En la introducción mostramos algunas propiedades básicas de las
matrices, y su capacidad de expresar sistemas de ecuaciones lineales.
A continuación mostraremos más propiedades importantes de matrices.

\subsection{Renglones y columnas}
La matriz renglón $i$ de la matriz $A$ la denotamos
$A_{i,.}$, y la columna $j$ por $A_{.,j}$.

\subsection{Teoremas de multiplicación de matrices}
\subsubsection*{Nota}
Con la notación anterior de renglones y columnas, podemos
escribir el elemento $(i,j)$ de $AB$ como
$$A_{i,.} B_{.,j}.$$

\subsubsection{Columna o renglón de una multiplicación}
La columna $j$ de la multiplicación de $AB$ es igual a la
multiplicación de $A$ por la columna $j$ de $B$. Es decir
$(AB)_{.,j} = A B_{.,j}$. Análogamente se tiene lo mismo para renglones,
$(AB)_{i,.} = A_{i,.} B$.

\subsubsection*{Prueba}
El elemento $i$ de $(AB)_{.,j}$ es $A_{i,.} B_{.,j}$.
El elemento $i$ de $AB_{.,j}$ es igualmente, $A_{i,.}B_{.,j}$.
Análogamente para renglones. \hfill ////

\subsubsection{La multiplicación de una matriz por un vector (columna)}
Sea $A\in F^{m\times n}$ y $x=(x_1,\ldots,x_n)^t\in F^{n\times 1}$,
entonces
$$Ax=\sum^n_{j=1}A_{.,j} x_j.$$
Es decir, $Ax$ es la combinación lineal de las columnas de $A$
con los elementos de $x$ como coeficientes.
\subsubsection*{Prueba}
El elemento $i$ de $Ax$ es
$$A_{i,.}x=\sum^n_{j=1}A_{i,j}x_j$$
mientras que el elemento $i$ de $\sum^n_{j=1}A_{.,j} x_j$
es claramente también
$$\sum^n_{j=1}A_{i,j} x_j.$$
\hfill ////

\subsubsection{Multiplicación de matrices como combinación lineal
de columnas o renglones}
\begin{enumerate}
    \item La columna $j$ de $AB$ es la combinación lineal de columnas
    de $A$ con los elementos de la columna $j$ de $B$ como coeficientes.
    \item El renglón $i$ de $AB$ es la combinación lineal de renglones
    de $B$ con los elementos del renglón $i$ de $A$ como coeficientes.
\end{enumerate}
\subsubsection*{Prueba}
1. $(AB)_{.,j}$ es $AB_{.,j}$ como ya probamos, a su vez esto
es un producto de matriz por columna, entonces es la combinación lineal
de columnas de $A$ con los elementos de la columna $B_{.,j}$ como
coeficientes.

2. $(AB)_{i,.}$ es $A_{i,.}B$. Su transpuesta es $B^t{A_{i,.}}^t$, donde
${A_{i,.}}^t$ es una columna, por lo tanto $B^t{A_{i,.}}^t$ es la combinación lineal de columnas de $B^t$ con
elementos de ${A_{i,.}}^t$ como coeficientes. La transpuesta de la
transpuesta, es el renglón original $(AB)_{i,.}$. Entonces
$(AB)_{i,.}$ es la combinación lineal de renglones de $B$ (cuyas transpuestas eran las columnas en $B^t$) con coeficientes del renglón $A_{i,.}$.
\hfill ////

\subsection{Factorización columna-renglón y rango de una matriz}
\subsubsection{Rango de columnas y renglones}
El rango de las columnas de una matriz $A\in F^{m\times n}$
es la dimensión de \\
$\spn \{A_{.,j}\}_{j=1}^n$ es decir, del
espacio generado por las columnas de $A$.

Análogamente se define el rango de renglones de $A$ como la 
dimensión del espacio generado por los renglones de $A$.

\subsubsection{Factorización columna-renglón}
Sea $A\in F^{m\times n}$ con rango de columnas $p\geq 1$,
entonces existen $C\in F^{m\times p}$ y $R\in F^{p\times n}$
tal que $A=CR$.
\subsubsection*{Prueba}
Sea $C$ la matriz cuyas columnas corresponden a una base del
espacio de columnas de $A$, como el rango de columnas de $A$ es $p$,
$C$ es $m\times p$.

Cada columna $j$ de $A$ es una combinación lineal
de columnas de $C$ pues estas son básicas para el espacio de columnas
de $A$. Sean los coeficientes de tal combinación lineal, los elementos
de las columna $j$ de la matriz $R$, entonces $R$ es $p\times n$.

Por sus dimensiones existe $CR$ de dimensión $m\times n$, cuya columna
$j$ es $CR_{.,j}$, luego esta columna es combinación lineal de columnas
de $C$ con coeficientes de la columna $j$ de $R$, que como dijimos, son
exactamente los coeficientes para que tal combinación lineal sea la columna $j$ de $A$. \hfill ////

\subsubsection{Rango de columnas es el rango de renglones}
Si $A\in F^{m\times n}$, su rango de columnas es igual a su rango
de renglones.
\subsubsection*{Prueba}
Sea $p\geq 1$ el rango de columnas de $A$, y factorizamos $A$
por columna-renglón $CR$. Como cada renglón de $A$ es
combinación lineal de los renglones de $R$, y $R$ tiene $p$
renglones, el rango de renglones de $A$ es a lo más $p$.
Es decir, el rango de renglones es menor o igual que el rango de 
columnas.

De la misma manera si $q\geq 1$ es el rango de renglones, en $A^t$
este es su rango de columnas. Y por el argumento anterior, su
rango de renglones es menor o igual que su rango de columnas.

En la matriz original, esto es que el rango de columnas es menor
o igual que el rango de renglones. Probando para el caso que el rango 
de columnas y el rango de renglones es al menos $1$, estos son
iguales.

Si el rango de columnas es $0$, este es el caso en que el espacio de
columnas no tiene base, pero es generado por $\{0\}$, de manera que
todas las columnas son $0$, y el rango de renglones también es $0$. Análogamente cuando el rango de renglones es $0$.
\hfill ////

\subsubsection{Rango}
Finalmente, podemos definir el rango de una matriz como el rango de
columnas, pues este es igual al rango de renglones. 
\newpage

\section{Relación de matrices y mapeos lineales}
\subsection{Matriz de un mapeo lineal}
Dado un mapeo lineal $T: V\rightarrow W$ y bases $B=\{b_i\}^n_{i=1}$
$C=\{c_j\}^m_{j=1}$, definimos la matriz $A\in F^{m\times n}$ tal que
$$Tb_j = \sum^m_{i=1} A_{i,j} c_i$$
para $j=1,\ldots,n$.

\newcommand{\M}{\mathcal{M}}
Dicho de otra forma, los elementos de $A_{.,j}$ son los coeficientes
de $Tb_j$ expandido en la base $C$ de $W$. Y denotamos a $A$, por
$\mathcal{M}(T,B,C)$, o cuando no haya confusión con las bases,
$\mathcal{M}(T)$.

\subsubsection{Ejemplo}
Haremos notar, que encontrar la matriz correspondiente a un mapeo lineal
es un cuanto confuso, pues al expandir un vector del contradominio,
tenemos una suma donde los coeficientes de la columna se encuentran
en forma horizontal, sugiriendo erróneamente que corresponden a un
renglón.

Sea $T:F^3\rightarrow F^2$ dada con las bases estándar para el dominio y
contradominio por $T(x,y,z)^t = (x,y + z)^t$, tenemos:
{
$$T\begin{bmatrix}
    1\\
    0\\
    0    
\end{bmatrix}
 = 
 1\begin{bmatrix}
    1\\
    0
 \end{bmatrix}
 +0\begin{bmatrix}
    0\\
    1
 \end{bmatrix}
 $$
 $$T\begin{bmatrix}
    0\\
    1\\
    0
 \end{bmatrix}
 =
 0\begin{bmatrix}
    1\\
    0
 \end{bmatrix}
 +1\begin{bmatrix}
    0\\
    1
 \end{bmatrix}
 $$
 $$
 T\begin{bmatrix}
    0\\
    0\\
    1
 \end{bmatrix}
 =
 0\begin{bmatrix}
    1\\
    0
 \end{bmatrix}
 +
 1\begin{bmatrix}
    0\\
    1
 \end{bmatrix}
 .$$
}
De manera que
$$\mathcal{M}(T) = \begin{bmatrix}
    1 & 0 & 0\\
    0 & 1 & 1
\end{bmatrix}.$$
\newpage

\subsection{Matriz de la multiplicación de mapeos lineales}
Sean $B$, $D$, y $C$ bases para $V$, $U$, y $W$, $T\in\mathcal{L}(V,U)$,
$S\in\mathcal{L}(U,W)$, entonces
$$\M(ST,B,C) = \M(S,D,C)\M(T,B,D).$$
\subsubsection*{Prueba}
Sean $v_i$, $u_k$ y $w_j$ elementos de las bases $B$, $D$, y $C$ respectivamente, $X=\M(S,D,C)$, y $Y=\M(T,B,D)$.

$\M(ST,B,C)$ está dada por: 
\begin{align*}
    STv_i &= S\left(\sum_k X_{i,k}u_k\right)\\
    &= \sum_k X_{i,k} Su_k\\
    &= \sum_k X_{i,k} \sum_j Y_{k,j} w_j \\
    &= \sum_j \left(\sum_k X_{i,k} Y_{k,j}\right) w_j.
\end{align*}
Donde $\sum_k X_{i,k} Y_{k,j} = (\M(S,D,C)\M(T,B,D))_{i,j}$.
De aquí que la matriz de $ST$ es la matriz de $S$ por la matriz
de $T$ de acuerdo a sus respectivas bases.

\hfill ////

\subsection{El espacio de mapeos lineales y de matrices son isomorfos}
Sea $B=\{v_i\}_{i=1}^n$ una base para $V$ y $C=\{w_j\}_{j=1}^m$ una base
para $W$. Entonces $\mathcal{M}(\cdot, B, C)$ (que denotaremos por 
$\mathcal{M}$) es un isomorfismo
entre $\mathcal{L}(V,W)$, y $F^{m\times n}$.
\subsubsection*{Prueba}
Es trivial que $\M$ es lineal, solo falta probar que es inyectivo y 
sobreyectivo.

Sea $T\in\ker\M$, es decir $\M T$ es la matriz $0$. Luego por la
definición de $\M T$, $Tv_i = 0$. Por lo que en general $Tx=0$,
y $T=0$. Entonces $\dim\ker\M = 0$, y $\M$ es inyectivo.

Sea $A\in F^{m\times n}$, existe $T\in\mathcal{L}(V,W)$ definido
por $$Tv_j = \sum^m_{i=1}A_{i,j} w_j.$$
De manera que $A=\M T$ y por lo tanto $\M$ es sobreyectivo.
\hfill ////

\subsubsection{Corolario: Dimensión del espacio de mapeos lineales}
$\mathcal{L}(V,W)$ con $V$ de dimensión $n$ y $W$ de dimensión $m$
ambas finitas, tiene dimensión $nm$ por virtud de que es isomorfo a
$F^{n\times m}$. Dicho independientemente de las bases
$$\dim \mathcal{L}(V,W) = \dim V \times \dim W.$$

\subsection{La matriz de un vector}
Sea $\{v_i\}_{i=1}^n$ base de $V$ y $x=\sum_i\alpha_i v_i$, 
entonces se define la matriz de $x$ con esta base por la matriz
$n\times 1$ por
$$\M x = \begin{bmatrix}
    \alpha_1 \\
    \vdots \\
    \alpha_n
\end{bmatrix}.$$

\subsection{La columna de la matriz de una transformación}
Sea $\mathcal{L}(V,W)$ donde $v_1$ hasta $v_n$ son una base para $V$
y $w_1$ a $w_m$ una base para $W$, entonces
$$\M(T)_{.,j} = \M (Tv_j).$$
\subsubsection*{Prueba}
Sea $A=\M T$, tenemos
$$Tv_j = \sum^m_{i=1}A_{i,j} w_i$$
$$\M(Tv_j) = \begin{bmatrix}
    A_{1,j}\\
    \vdots\\
    A_{m,j}
\end{bmatrix}
=
A_{.,j}.$$
\hfill ////

\subsection{Mapeos lineales actúan como matrices}
Sea $T$ un mapeo lineal con dominio de base $v_j$ y dimensión $n$,
contradominio de base $w_i$ y dimensión $m$, entonces
$$\M (Tx) = \M (T)\M (x).$$
\subsubsection*{Prueba}
Primero, sea $x$ expresado en la base $v_j$ con coeficientes $\alpha_j$,
entonces $$Tx =\sum_j\alpha_j Tv_j$$
luego $$\M(Tx)=\sum_j\alpha_j\M(Tv_j)$$
porque $\M$ es lineal. Como probamos antes
$M(Tv_i) = M(T)_{.,i}$, entonces
\begin{align*}
    \M(Tx) &= \sum_j\alpha_j\M(T)_{.,j}\\
    &= \M(T) \begin{bmatrix}
        \alpha_1 \\
        \vdots \\
        \alpha_n
    \end{bmatrix}\\
    &= \M(T)\M(x).
\end{align*}
Donde la segunda igualdad está dada porque la combinación
lineal de columnas de una matriz es la multiplicación de una matriz
por la columna con los coeficientes de la combinación lineal por 
elementos. \hfill ////

\subsubsection{Nota}
Recordemos que $\M(T)$ depende de las bases del dominio y 
contradominio de $T$. Mientras que $M(x)$ depende de la base
de $V$ donde está $x$.

\subsection{El rango de una matriz de un mapeo lineal}
Sea $T$ un mapeo lineal con dominio y contradominio finito,
entonces para cualquier selección de bases $\M(T)$ tiene rango
igual a $\dim\im T$.
\subsubsection*{Prueba}
El mapeo lineal $y \mapsto \M y$ con $y\in\im T$ es un isomorfismo
entre $\im T$ y el espacio de columnas de $\M(T)$, pues
$\M y=\M(Tx) = \M(T)\M(x)$ para algún $x\in V$. Y como ya mostramos,
tenemos que $\M(T)\M(x)$ está en el espacio de columnas de $\M(T)$.
Sigue que $\dim\im T$ es el rango de columnas de $\M(T)$.
\hfill ////

\subsection{La matriz identidad}
$I_n$ (o simplemente $I$ si no hay confusión) es la matriz identidad
de dimensiones $n\times n$ definida por $I_{i,j}=\delta_{i,j}$
donde $\delta_{i,j}$ es $0$ si $i\neq j$ y $1$ de otra manera.

\subsection{Inversa de una matriz}
Se dice que una matriz $A$ $m\times n$ es invertible si
existe una matriz $B$ $n\times m$ tal que $AB=I_m$ y
$BA=I_n$.

\subsubsection{Propiedades de una matriz invertible}
Análogamente a las transformaciones invertibles:
\begin{enumerate}
    \item Una matriz invertible es cuadrada.
    \item Su inversa es única y denotada por $A^{-1}$.
    \item Si $A$ y $B$ son invertibles de la misma dimensión,
    $AB$ es invertible y $(AB)^{-1}=B^{-1}A^{-1}$.
\end{enumerate}
\subsubsection*{Prueba}
1. Sea $A$ $m\times n$ invertible con $B$ como en la definición, y sean
$T= x\mapsto Ax$ y $S= y\mapsto By$ con $x\in F^n$ y $y\in F^m$.
Tenemos
\begin{align*}
    (TS)y &= ABy = y\\
    (ST)x &= BAx = x.
\end{align*}
Entonces $S=T^{-1}$, de manera que $T$ es un isomorfismo entre
$F^n$ y $F^m$, por lo tanto $n=m$.

2. Como se tiene que $S=T^{-1}$ es único para $T$, $B=\M(S)$ también
es único para $A$.

3. Sean $S$ y $T$ como en 1, por la proposición de la inversa
de la multiplicación de mapeos lineales, sigue que
$(AB)^{-1} = B^{-1}A^{-1}$.
\hfill ////

\subsection{Matriz de la identidad en distintas bases}
Sean $B$ y $C$ distintas bases de $V$, entonces
$\M(I,B,C)$ es la inversa de\\
$\M(I,C,B)$.
\subsubsection*{Prueba}
Por como es la matriz de una multiplicación de transformaciones,
tomando en cuenta sus bases, tenemos
$$\M(I)=\M(II,C,C) = \M(I,B,C) \M(I,C,B)$$
y
$$\M(I)=\M(II,B,B) = \M(I,C,B) \M(I,B,C).$$
Donde la primer igualdad en cada ecuación viene de que la matriz
de la identidad de una base a la misma base es meramente la matriz
identidad. \hfill ////
\newpage

\subsection{Matriz de cambio de bases}
Sea $T\in\mathcal{L}(V)$, y $v_i$, $u_i$ bases de $V$.
Entonces si $A=\M(T,v_i)$, $B=\M(T,u_i)$
entonces
$$A=\M(I,u_i, v_i)B\M(I,v_i,u_i).$$
\subsubsection*{Prueba}
Simplemente se factoriza $A$ usando que $T=TI$,
luego se factoriza el primer factor resultante $\M(T, u_i, v_i)$
usando que $T=IT$.
\begin{align*}
    A&=\M(TI,v_i)\\
    &=\M(T, u_i, v_i)\M(I, v_i, u_i)\\
    &=\M(IT, u_i, v_i)\M(I, v_i, u_i)\\
    &=\left(\M(I, u_i, v_i)\M(T,u_i)\right)\M(I, v_i, u_i)\\
    &=\M(I, u_i, v_i)B\M(I, v_i, u_i).
\end{align*}
\hfill ////

\subsection{La matriz de la inversa de un operador lineal}
Sea $T$ un operador lineal invertible en $V$, con base $v_i$ y
dimensión $n$. Entonces $\M(T^{-1}) = M(T)^{-1}$.
\subsubsection*{Prueba}
\begin{align*}
    \M(I)&=\M(TT^{-1})\\
    &=\M(T)\M(T^{-1}).
\end{align*}
De donde multiplicando por $M(T)^{-1}$ por la izquierda,
directamente sigue el resultado.\hfill ////

\subsection{Invertibilidad de los factores de un operador lineal}
Sean $S$ y $T$ operadores lineales sobre $V$ de dimensión finita,
entonces $ST$ es invertible ssi $S$ y $T$ son invertible.
\subsubsection*{Prueba}
Ya probamos $ST$ invertible $\impliedby$ $S$ y $T$ invertibles.

Ahora supongamos que $ST$ es invertible.

\textbf{$S$ es sobreyectiva:} Si $S$ no fuese sobreyectiva, $ST$ tampoco lo sería y tampoco sería
invertible. Por lo tanto $S$ es sobreyectiva.

\textbf{$T$ es inyectiva:} Si $T$ no fuese inyectiva, existen $x\neq y$ tal que $Tx=Ty$,
luego $STx=STy$, y $ST$ no sería inyectiva ni invertible. Por lo
tanto $T$ es inyectiva.

Como el espacio $V$ es de dimensión finita y los mapeos lineales son 
operadores, sobreyectividad o inyectividad implican invertibilidad, así 
$S$ y $T$ son invertibles.
\hfill ////
\newpage

\section{Espacios cocientes}
\subsection{Traslación de un subconjunto}
Definimos la traslación de un conjunto $U\subset V$ por $x\in V$ como
el conjunto $x+U:=\{x+u: u\in U\}$.

Llamamos a $x$ en $x+U$ el representante de la traslación. Puede haber distintos representantes que generen la misma traslación.

\subsection{Definición de espacio cociente}
Si $U$ es un subespacio de $V$, definimos el espacio cociente
$V/U$ como el conjunto de todas las traslaciones de $U$:
$$V/U=\{x+U:x\in V\}.$$

\subsection{Una traslación es igual a otra o ajenos}\label{traslaciones}
Sean $v$ y $w$ en $V$, y $U$ un subespacio de $V$. Entonces
$$v-w\in U \iff v+U = w+V \iff (v+U)\cap(w+U)\neq\emptyset.$$
\subsubsection*{Prueba}
La demostración es un ejercicio elemental de álgebra y conjuntos.

De aquí en adelante $U$ es un subespacio de $V$, y los elementos en
minúsculas son elementos cualesquiera de $V$
\subsection{Equivalencia}
Como acabamos de demostrar $v-w\in U \iff v+U=w+U$,
así que definimos la relación de equivalencia $\equiv$ por
$v\equiv w \iff v-w\in U$.

Queremos formar un espacio vectorial sobre el espacio cociente.
Para esto tenemos que definir correctamente la suma y multiplicación 
por escalar.

\subsection{Suma}
Sean $v\equiv x$ y $w\equiv y$, entonces $v+w \equiv x+y$.
\subsubsection*{Prueba}
Por definición tenemos $v-x\in U$ y $w-y\in U$.
Luego
\begin{align*}
    (v+w) - (x+y) &= (v-x) + (w-y) \in U
\end{align*} 
por cerradura del subespacio.\hfill ////

Esto demuestra que la suma de traslaciones no depende de su
representante.

De manera que definimos $(x+U)+(y+U) := (x+y)+U$ para \\
cualquier $x,y$.


\subsection{Multiplicación escalar}
Sea $v\equiv x$ y $\lambda$ un escalar, entonces
$\lambda v\equiv \lambda x$.
\subsubsection*{Prueba}
$v-x\in U$ entonces
$$\lambda v -\lambda x = \lambda(v-x) \in U$$
por cerradura de $U$. \hfill ////

Así que definimos $\lambda (v + U):= \lambda v + U$.

Ambas operaciones hacen que $V/U$ sea un espacio vectorial.
Hacemos notar que $0+U$ es $0$ en este espacio vectorial.

\subsection{Mapeo cociente}
El mapeo cociente $\pi: V \rightarrow V/U$ está definido por
$$\pi(x)=x+U$$
y es lineal.

\subsection{Dimensión del espacio cociente}
Si $V$ es de dimensión finita,
$$\dim V/U = \dim V - \dim U.$$
\subsubsection*{Prueba}
Tenemos que $x\in\ker \pi \iff x+U = 0+U \iff x\in U$, donde la
segunda equivalencia está dada por \ref{traslaciones}. De
manera que $\ker\pi = U$. Como $\pi: V \rightarrow V/U$ y $V$ es de 
dimensión finita, por el teorema fundamental de mapeos lineales
$$\dim V = \dim\ker\pi + \dim\im\pi$$
como $\im\pi=V/U$ y $\ker\pi = U$ sigue el resultado esperado
$$\dim V/U = \dim V - \dim U.$$
\hfill ////

Para la siguiente definición hacemos notar que para
$T\in\mathcal{L}(V,W)$ si $x+\ker T = y+\ker T$, por \ref{traslaciones}
tenemos $x-y\in\ker T$ y por lo tanto $Tx=Ty$.
\subsection{Mapeo inducido por el kernel de un mapeo lineal}
Dado $T$ definimos $\widetilde T: V/\ker T\rightarrow W$ por
$$\widetilde T(x+\ker T) = Tx$$
y notamos que $\widetilde T$ es lineal.

Una propiedad se hace inmediatamente aparente.
$$x\in\ker T \iff x+\ker T\in \ker\widetilde{T}.$$
La suficiencia viene de \ref{traslaciones}.
La necesidad, porque si $x+\ker T\in \ker\widetilde{T}$, entonces
$$\widetilde T(x+\ker T) = Tx = 0.$$

De la necesidad tenemos que $x+\ker T\in \ker\widetilde{T}$
implica que $x\in\ker T$, es decir $x +\ker T = 0+\ker T$. De manera
que $\ker\widetilde{T} = \{0+\ker T\}$, y $\widetilde{T}$ es inyectiva.

\subsubsection{Más propiedades de este mapeo}
Sea $T\in\mathcal{L}(V,W)$ entonces
\begin{enumerate}
    \item $\widetilde{T}\circ\pi = T$, donde
    $\pi:V/\ker T\rightarrow W$ es el mapeo cociente.
    \item $\widetilde{T}$ es inyectiva.
    \item $\im \widetilde{T} = \im T$.
\end{enumerate}
\subsubsection*{Prueba}
\begin{enumerate}
    \item $(\widetilde{T}\circ\pi)x = \widetilde{T}(x+\ker T) = Tx$
    \item Lo probamos anteriormente.
    \item Por definición.
\end{enumerate}
\hfill ////

Por consecuencia de punto 2 y 3, $V/\ker T$ y $\im T$ son
isomorfos.
\newpage

\section{Dualidad}
\subsection{Funcional}
Un funcional lineal en $V$, o simplemente funcional por el contexto de 
álgebra lineal, es un mapeo lineal $V$ a su campo.
Es decir, un elemento de $\mathcal{L}(V,F)$.

\subsection{Espacio dual}
El espacio dual de $V$ es definido por $\mathcal{L}(V,F)$.
Es decir, es el espacio de funcionales en $V$.

\subsection{Dimensión del espacio dual}
$\dim V' = \dim V$.
\subsubsection*{Prueba}
$\dim V' = \dim \mathcal{L}(V,F) = \dim V \dim F = \dim V$.

\subsection{Base dual}
Sea $v_j$ una base para $V$ de dimensión finita, la base dual de $v_j$ 
son los elementos $f_j\in V'$ donde 
$$f_j(v_i) = \delta_{i,j}$$
es decir, $f_j(v_i)=1$ para $i=j$ y $0$ de otra forma.

\subsubsection{Ejemplo}
Sea $f_j$ el funcional en $F^n$ que escoge la coordenada $j$
del vector $(x_i)$, es decir
$$f_j((x_i))=x_j.$$
Si $e_i$ es la base estándar de $F^n$
$$f_j(e_i)=\delta_{i,j}$$
por lo que $f_j$ forma la base dual de $e_i$.

\subsubsection{Propiedad inmediata}
Como los elementos de la base dual son lineales tenemos
para $x=\sum_i\alpha_i v_i$ donde $v_i$ es una base para $V$,
$$f_j(x)=\sum_i\alpha_i f_j(v_i) = \alpha_j.$$
De manera que en general, los elementos $f_j$ de la base dual
seleccionan la coordenada $j$ de $x$ con respecto a $v_i$. Por lo que
el ejemplo anterior no era una coincidencia.

De esto, tenemos directamente siguiente resultado.

\subsection{Los funcionales de la base dual mapean a coordenadas de un vector}
Sea $x\in V$ con $v_i$ como base y $f_j$ la base
dual, entonces
$$x = \sum_i f_i(x) v_i.$$

\subsection{La base dual es una base para el espacio dual}
Sea $V$ de dimensión finita. Entonces la base dual es una base para
$V'$.
\subsubsection*{Prueba}
Sea $v_i$ una base para $V$, $f_i$ la base dual.

Si $\sum_i\alpha_i f_i = 0$, es decir para cualquier
$x=\sum_j\beta_j v_j\in V$ se tiene
$$\sum_i\alpha_i f_i(x) = \sum_i\alpha_i \beta_i = 0.$$
En especial si se escoge $x=v_k$,
$$\sum_i\alpha_i f_i(v_k) = \alpha_k = 0.$$
De manera que $f_i$ son linealmente independientes.

La dimensión de $V'$ es igual al número de vectores $f_i$, y por
ser linealmente independientes, por \ref{bases}, son base de $V'$.
\hfill ////

\subsection{El mapeo dual}
Sea $T\in\mathcal{L}(V,W)$. El mapeo dual $T'\in\mathcal{L}(W',V')$,
está definido por
$$T'(f) = f\circ T$$
para todo $f\in W'$.

Por ser composición de mapeos lineales, $T'(f)$ es lineal,
y siguiendo el mapeo de espacios,
$T'(f):V\rightarrow W\rightarrow F$, se tiene que $T'(f)\in V'$.

\subsubsection{Ejemplo}
Sea $D\in\mathcal{L}(P(R))$ el operador de diferenciación
en el espacio de polinomios. Y sea $f$ el funcional dado por
$f(p) = p(3)$.

Entonces \begin{align*}
    D'(f)(p)&=(f\circ D)(p) \\
    &= f(Dp)\\
    &= (Dp)(3).
\end{align*}

\subsection{Propiedades algebraicas del mapeo dual}
Sean $T\in\mathcal{L}(V,W)$ y $\lambda\in F$. Entonces
\begin{enumerate}
    \item $(S+T)' = S' + T'$ para $S\in\mathcal{L}(V,W)$.
    \item $(\lambda T)' = \lambda T'$.
    \item $(ST)'=T'S'$ para $S\in\mathcal{L}(W,U)$.
\end{enumerate}
\subsubsection*{Proof}
Punto 1 y 2 son computaciones directas.

Sea $f\in U'$ (el dominio de $(ST)'$). Entonces
$$(ST)'(f) = f\circ (ST) = (f\circ S)T.$$
Mientras que $T'S'$ también tiene dominio $U'$, y podemos usar la
misma $f\in U'$ en $T'S'$:
$$(T'S')(f)=T'(S'(f))=T'(f\circ S)=(f\circ S)\circ T= (f\circ S)T.$$
\hfill ////

\subsection{Kernel e imagen del mapeo dual}
\subsubsection{Aniquilador de un subconjunto}
Para $U\subset V$, el aniquilador $U^0\subset V'$ de $U$ se define por
$$U^0 = \{f\in V': f(U) = \{0\}\}.$$

\subsubsection{Ejemplo}
Sea $e_1,e_2$ la base estándar para los pares de reales, $R^2$, y sea
$f_1,f_2$ la base dual.

Sea $U=\spn\{e_1\}$, sus elementos son de la forma $(\alpha,0)$.
Un funcional $f\in (R^2)'$ está dado por
$$f=\lambda_1 f_1 + \lambda_2 f_2.$$

Entonces para $u\in U$
$$f(u)=\lambda_1 f_1(u) + \lambda_2 f_2(u) = \lambda_1\alpha.$$

De manera que los funcionales de $U$ que mapean a $0$ tienen 
$\lambda_1=0$. Luego $U^0$ consiste de los $f=\lambda f_2$.
\newpage

\subsubsection{Los aniquiladores son subespacios del espacio dual}
Sea $U\subset V$, entonces $U^0$ es un subespacio de $V'$.
\subsubsection*{Prueba}
Es trivial que $0\in V'$ está en $U$.

Sea $f,g\in U^0$ y $\lambda\in F$. Para $u\in U$ tenemos
$$(f+\lambda g)(u)=f(u)+\lambda g(u) = 0$$
por lo tanto $f+\lambda g \in U^0$.\hfill ////

\subsubsection{Dimensión del aniquilador}
Sea $V$ de dimensión finita y $U$ un subespacio, entonces
$$\dim V = \dim U^0 + \dim U.$$
\subsubsection*{Prueba}
Sea $I$ en $\mathcal{L}(U, V)$ definido por $I(u)=u$
para $u\in U$. Entonces $I'\in\mathcal{L}(V', U')$.

Por el teorema fundamental de
mapeos lineales tenemos
\begin{align}
    \dim V' &= \dim\ker I' + \dim\im I'.
\end{align}
Sustituiremos términos de esta ecuación por términos
equivalentes.

1. $\ker I' = U^0$:
\begin{align*}
    \ker I' &= \{f\in V': I'(f)=0\}\\
    &= \{f\in V': f\circ I=0\}\\
    &= \{f\in V': f(U)=\{0\}\}\\
    &= U^0.
\end{align*}

2. $\im I' = U'$:
\begin{align*}
    \im I' &= I'(V')\\
    &=\{f\circ I: f\in V'\}\\
    &=\{f: f\in U'\}\\
    &=U'.
\end{align*}

Sustituyendo con los puntos 1 y 2 en (1), y usando que
$\dim V'=\dim V$ y $\dim U'= \dim U$, tenemos
$$\dim V = \dim U^0 + \dim U.$$
\hfill ////

Lo siguiente es consecuencia directa del teorema anterior.
\subsubsection{Condiciones que determinan el aniquilador}
\label{aniq}
Sea $V$ de dimensión finita y $U$ un subespacio.
\begin{enumerate}
    \item $U^0 = \{0\} \iff U=V$
    \item $U^0 = V' \iff U=\{0\}$
\end{enumerate}

\subsection{El kernel de la transformación dual}
\label{kerdual}
Sea $T\in\mathcal{L}(V,W)$.
\begin{enumerate}
    \item $\ker T' = (\im T)^0$.
    \item Si $V$ y $W$ son de dimensión finita,
    $\dim\ker T' = \dim\ker T + \dim W - \dim V$.
\end{enumerate}
\subsubsection*{Prueba}
1. Sea $f\in\ker T'$, entonces $T'(f) = f\circ T = 0$.
Esto quiere decir que $f(\im T) =\{0\}$, por definición
$f\in (\im T)^0$. De manera que $\ker T' \subset (\im T)^0$.

Como $T$ es sobreyectiva limitando $W$ a $\im T$,
existe siempre $v\in V$ tal que $w= Tv$ para $w\in W$. Entonces 
$f\in (\im T)^0$ es equivalente a que $f\circ T = 0$, luego
$f\in\ker T'$.

2. Tenemos
\begin{align*}
    \dim\ker T' &= \dim (\im T)^0\\
    &= \dim W - \dim \im T\\
    &= \dim W - (\dim V - \dim\ker T).
\end{align*}
Donde la primer igualdad viene dada por el punto 1,
la segunda por el teorema de dimensión del aniquilador, 
y la última por el teorema fundamental de mapeos lineales.
\hfill ////

\subsection{El dual de un mapeo es inyectivo ssi el mapeo es sobreyectivo}
Si $T\in\mathcal{L}(V,W)$ con $V,W$ finitos, entonces
$T$ sobreyectivo si y sólo si $T'$ es inyectivo.
\subsubsection*{Prueba}
Tenemos
\begin{align*}
    T\text{ sobreyectivo} &\iff \im T = W\\
    &\iff (\im T)^0 = \{0\}\\
    &\iff \ker T' = \{0\}.
\end{align*}
Donde la segunda equivalencia viene de \ref{aniq} y la tercera
de \ref{kerdual}.\hfill ////

\subsection{El rango del mapeo dual}
Si $V$ y $W$ son de dimensión finita y $T$ un mapeo lineal
entre ellos, entonces
\begin{enumerate}
    \item $\dim\im T' = \dim\im T$
    \item $\im T = (\ker T)^0$.
\end{enumerate}

\subsection{El mapeo dual es sobreyectivo ssi el mapeo es inyectivo}
Si $V$ y $W$ son de dimensión finita y $T\in\mathcal{L}(V,W)$
entonces
$$T\text{ inyectivo}\iff T'\text{ sobreyectivo}.$$

\subsection{La matriz del mapeo dual}
Si $V$ y $W$ son de dimensión finita y $T\in\mathcal{L}(V,W)$.
Entonces
$$\M(T') = (\M(T))^t.$$
\subsubsection*{Prueba}
Sean $v_\alpha$, $w_\beta$, bases para $V$ y $W$, y $f_i$, $g_j$ sus
bases duales respectivamente.
Notemos que $\alpha$ e $i$ recorren los mismos índices al igual que
$\beta$ y $j$.
Así, quedan implicadas qué bases usamos para
$A=\M(T)$ y $B=M(T')$. Comparemos sus términos.

\begin{align*}
    T'g_j &= \sum_i B_{i,j} f_i
\end{align*}

Evaluando en $v_\alpha$ tenemos
\begin{align*}
    (T'g_j)v_\alpha &= \sum_i B_{i,j} f_i(v_\alpha)\\
    &= B_{\alpha,j}.
\end{align*}
Expresando lo mismo en términos de la definición de $T'$:
\begin{align*}
    (T'g_j)v_\alpha &= g_j(Tv_\alpha)\\
    &= g_j\left(\sum_\beta A_{\beta,\alpha} w_\beta\right)\\
    &=\sum_\beta A_{\beta,\alpha} g_j(w_\beta)\\
    &= A_{j,\alpha}.
\end{align*}
$$\implies B_{i,j} = A_{j,i}.$$
\hfill ////

Con la teoría desarrollada sobre dualidad, podemos dar una prueba
sobre el rango de una matriz usando solo igualdades.
\subsection{El rango de columnas es igual al rango de renglones}
Sea $A\in F^{m\times n}$, entonces su rango de columnas es
igual a su rango de renglones.
\subsubsection*{Prueba}
Definimos $T: F^n\rightarrow F^m$ por $Tx=Ax$ de manera que
$\M(T)=A$.
\begin{align*}
    \text{rango de columnas de $A$} &= \dim\im T\\
    &= \dim\im T'\\
    &= \text{rango de columnas de $A^t$}\\
    &= \text{rango de renglones de $A$}.
\end{align*}
\hfill ////
\newpage

\section{Polinomios}
En vez de usar determinantes para la sección de eigenvalores y
eigenvectores, Axler usa el concepto de polinomio mínimo de un
operador lineal. Así que será útil repasar polinomios.
No mostraremos pruebas pero escribiremos resultados importantes.

Se supone familiaridad con los números complejos. $F$ se
refiere a los complejos o los reales.

\subsection{Ceros de polinomios}
Se recuerda que un polinomio de grado $m$
$p:F\rightarrow F$ es una función de la forma
\begin{align*}
    p(z) &= a_0+a_1 z+\ldots+a_n z^n\\
    &= \sum^n_{k=0} a_k z^k
\end{align*}
con $m\neq 0$ y $z\in F$.

\subsection{Ceros de un polinomio}
Un cero (o raíz) $\lambda$ de un polinomio $p$ es un escalar de 
$F$ que anula el polinomio, es decir
$$p(\lambda) = 0.$$

\subsection{Cada cero de un polinomio corresponde a un factor de grado uno}
Sea $p$ un polinomio de grado $n$, entonces
$p(\lambda) = 0$ si y sólo si existe un polinomio $q$
de grado $n-1$ tal que para todo $z$
$$p(z)=(z-\lambda)q(z).$$

\subsection{Algoritmo de división para polinomios}
Sean $p$ y $s$ polinomios con $s\neq 0$. Entonces existen
polinomios $q$, $r$ tal que
$$p=sq+r$$
con el grado de $r$ menor que $s$.

\subsection{Teorema fundamental del algebra, primer versión}
Todo polinomio no constante de coeficientes complejos, tiene
un cero en los complejos.

\subsection{Teorema fundamental del algebra, segunda versión}
\label{algebrafundamental}
Si $p$ es un polinomio sobre los complejos no constante de 
coeficientes complejos, de grado $n$,
entonces $p$ tiene una factorización única
$$p(z)=c\prod^n_{i=1}(z-\lambda_i)$$
con $c,\lambda_j$ en los complejos.

\subsection{Factorización de polinomios sobre los reales}
Si $p$ es un polinomio sobre los complejos con coeficientes 
reales y $\lambda$ es un cero de $p$, entonces $\overline\lambda$ también
lo es.

\subsection{Factorización de un polinomio cuadrático}
Sean $b,c$ reales, entonces el polinomio cuadrático de
coeficientes reales
$$p(x)=x^2+2bx+c$$
tiene factorización
$$(x-\lambda_1)(x-\lambda-2)$$
con $\lambda_i$ reales, si y sólo si
$b^2-4c\geq 0$.

\subsection{Factorización de un polinomio sobre los reales}
Sea $p$ un no constante polinomio sobre los reales, entonces
$p$ tiene una factorización única de la forma
$$p(x) = c\prod^m_{i=1}(x-\lambda_i)
\prod^M_{j=1}(x^2+2b_jx+c_j)$$
con $c,\lambda_i,b_j,c_j$ reales con $4b^2_j < 4c_j$.
\newpage

\section{Eigenvalores y eigenvectores}
\subsection{Operador}
Un mapeo lineal de un espacio vectorial al mismo espacio 
vectorial es llamado un operador (lineal).

Sean $T$ un operador en $V$ y, $V_i$ más de un subespacio
distintos de cero. Tenemos
$$T(x) = \sum_i T_{V_i}x_i$$
con $x_i\in V_i$, donde $T_{V_i}$ es $T$ restringido al
subespacio $V_i$. De manera que para entender $T$ dados
estos subespacios, solo necesitamos entender cada $T_{V_i}$.
Los cuales están definidos en un subespacio más pequeño que
$V$.

Sin embargo hay un problema, puede que $T_{V_i}$ no sea un
operador en $V_i$, es decir, que algunos $x\in V_i$
son mapeados fuera de $V_i$.

En esta sección también omitiremos algunas pruebas, no porque
sean triviales, al contrario, por su complejidad.

\subsection{Subespacio invariante}
Sea $T$ un operador en $V$, Un subespacio $U$ es invariante
bajo $T$ si $T(U)\subset U$.

De manera que $U$ es invariante bajo $T$ si $T_U$ es un
operador.

\subsubsection*{Subespacios invariantes unidimensionales}
Cambiamos nuestra atención a subespacios invariantes de una
dimensión. Un subespacio $U$ es invariante bajo $T$
de dimensión $1$
si y sólo si existe un vector $v\neq 0$ de $V$ tal que
$$U = \{\lambda v: \forall \lambda\} = \spn(v).$$

Si $U$ unidimensional es invariante bajo un operador $T$ de 
$V$, entonces $Tv\in U$ para $v\in U$, entonces hay un 
escalar $\lambda$ tal que 
$$Tv=\lambda v.$$
Recíprocamente, si $Tv=\lambda v$ para algún $\lambda$,
entonces $\spn(v)$ es un subespacio
unidimensional invariante bajo $T$.
\newpage

\subsection{Eigenvalor}
Sea $T$ un operador en $V$. Un escalar $\lambda$ es llamado
un eigenvalor si existe $v\neq 0$ tal que
$$Tv=\lambda v.$$

\subsection{Condiciones equivalentes para ser un eigenvalor}
Sea $T$ de dimensión finita, $T$ operador en $V$, y
$\lambda$ un escalar, entonces las siguientes proposiciones
 son equivalentes.
 \begin{enumerate}
    \item $\lambda$ es eigenvalor de $T$.
    \item $T-\lambda I$ no es inyectiva.
    \item $T -\lambda I$ no es sobreyectiva.
    \item $T -\lambda I$ no es invertible.
 \end{enumerate}
 \subsubsection*{Prueba}
 Tenemos que punto 1 es equivalente a punto 2, porque
 $Tv=\lambda v$ si y sólo si $(T-\lambda I)v = 0$ para todo
 $v\in V$. De manera que para $u\neq v$, $(T-\lambda I)v =
 (T-\lambda I)u$. Punto 2, 3, y 4 son equivalentes por
 un teorema de transformaciones que mapean el dominio
 a un contradominio de la misma dimensión.
 \hfill ////

 \subsection{Eigenvector}
Sea $T$ un operador en $V$ y $\lambda$ un eigenvalor
de $T$. Entonces un vector $v$ es llamado un
eigenvector de $V$ correspondiente a $\lambda$, si
$v\neq 0$ y $Tv=\lambda v$.

\subsection{Eigenvectores linealmente independientes}
Si $T$ es un operador lineal en $V$. Entonces la lista
de eigenvectores correspondiente a distintos eigenvalores
es linealmente independiente.

\subsection{Un operador no puede tener más eigenvalores que la dimensión de su espacio}
Si $V$ es de dimensión finita, un operador en $V$ no puede
tener más de $\dim V$ eigenvalores distintos. 
\subsubsection*{Prueba}
Sea $T$ un operador, si $m$ $\lambda_i$ son distintos 
eigenvalores, por el teorema anterior los eigenvectores
$v_i$ correspondientes son linealmente independientes.
Por teoría de espacios vectoriales de dimensión finita,
esta lista de eigenvectores no puede tener una dimensión
mayor a $\dim V$. \hfill ////

\subsection{Polinomios aplicados a los operadores}
\subsubsection{Notación de potencias de un operador}
Si $T$ es un operador en $V$, y $m$ un entero positivo.
\begin{enumerate}
    \item $T^0= I$.
    \item $T^1 = T$.
    \item $T^m$ es definido recursivamente como $T(T^{m-1})$.
    \item Si $T$ es invertible, se define $T^{-m}$ como
    $$(T^{-1})^m.$$
\end{enumerate}

\subsubsection{Notación de polinomio de un operador}
Si $T$ es un operador en $V$ y $p$ un polinomio sobre $F$
definido por $$p(z)=\sum^n_{k=0}a_k z^k$$
para $z\in F$, entonces definimos el operador $p(T)$ como
$$p(T)=\sum^n_{k=0}a_k T^k.$$

\subsubsection{La multiplicación de polinomios de operadores es conmutativa}
\subsubsection*{Prueba}
Es trivial que $pq=qp$ si $p$ y $q$ son polinomios sobre un
campo. Entonces $$p(T)q(T) = (pq)T = (qp)T = q(T)p(T).$$
\hfill ////

Es fácil comprobar que $\ker T$ y $\im T$ son invariantes bajo
$T$, además de los casos más triviales $\{0\}$ y $V$.
Pero además, los polinomios de $T$ conservan esta propiedad.
\newpage

\subsubsection{El kernel y la imagen de un polinomio de un operador son invariantes}
Sea $T$ un operador en $V$ y $p$ un polinomio sobre $F$. 
Entonces $\ker p(T)$ e $\im T$ son invariantes bajo $T$.
\subsubsection*{Prueba}
Supóngase que $u\in\ker p(T)$, mostramos que $Tu\in\ker p(T)$.
$$(p(T))(Tu)=(p(T)T)u=(Tp(T))u=T(p(T))u=T(p(T)u)=T(0)=0.$$
Donde la segunda igualdad viene de la conmutatividad de
polinomios, en este caso $T=i(T)$ donde $i$ es el polinomio
identidad $p(z)=z$, y la penúltima igualdad viene de la
suposición que $u\in\ker p(T)$.

\subsection{El polinomio mínimo}
\subsubsection{Existencia de eigenvalores}
Todo operador, en un espacio de dimensión finita distinto de cero
sobre los complejos, tiene un eigenvalor.
\subsubsection*{Prueba}
Sea $T$ un operador en $V$ de dimensión $n$. Tomando $v\neq 0$, la siguiente lista no es linealmente 
independiente:
\begin{align}\label{lista}
    v, Tv, \ldots, T^n v
\end{align}
pues tiene $n+1$ vectores. Es decir, que existe
una combinación lineal de esta lista que es igual a $0$,
con algún coeficiente distinto de $0$. De manera que se
puede representar esta combinación lineal con un polinomio
$p$, con la forma
\begin{align}
    p(T)v = 0.
\end{align}

Sean los coeficientes de la combinación lineal
$a_0,\ldots,a_n$ para $v,Tv,\ldots,T^n v$ respectivamente.
Si $p$ fuera constante, $a_0$ sería
el único coeficiente distinto de $0$, pero
esto implicaría que
$$p(T)v=a_0v =0$$
es falso, pues $v\neq 0$. De manera que tenemos que $p$
no es constante.

Existe un conjunto de polinomios $r$ con
coeficientes $a_0,\ldots,a_m$ para algún $m\leq n$, de 
manera que $r(T)v=0$ se cumple, y $r$ no es constante
(por lo menos, $m=n$).
Tomamos el $r$ con $m$ mínimo del conjunto.

Entonces por la primer versión del teorema fundamental del
álgebra \ref{algebrafundamental}, existen
$\lambda$ y $q$ tal que
$$r(z) = (z-\lambda)q(z).$$
Implicando que
$$r(T)v = (T-\lambda I)q(T)v = 0,$$
o
$$Tq(T)v = \lambda q(T)v.$$
Como $q$ es de menor grado que $r$, no cumple que $q(T)v = 0$ o
no cumple que $q$ no es constante, es decir, en ambos casos
$q(T)v\neq 0$.

Luego $\lambda$ es
eigenvalor de $T$ y $q(T)v$ un respectivo eigenvector.
\hfill ////

\subsection{Eigenvalores y el polinomio mínimo}
\subsubsection{Polinomio mónico}
Un polinomio mónico es un polinomio cuyo coeficiente de
mayor grado es $1$.

\subsubsection{Existencia y unicidad del polinomio mínimo}
Sea $T$ un operador en $V$. Entonces hay un polinomio mónico
$p$ único de grado mínimo tal que $p(T)=0$. Además
su grado es menor o igual a la dimensión de $V$.

\subsubsection{Polinomio mínimo}
Suponga que $T$ es un operador, sobre $V$ de dimensión finita.
Entonces el polinomio mínimo de $T$ es el único
polinomio mónico $p$, de grado mínimo tal que 
$p(T)=0$.

\subsubsection{Eigenvalores son los ceros del polinomio mínimo}
Sea $T$ un operador en $V$.
\begin{enumerate}
    \item Los ceros del polinomio mínimo son los eigenvalores
    de $T$.
    \item Si $V$ es un espacio vectorial complejo, entonces
    el polinomio mínimo de $T$ tiene la forma
    $$(z-\lambda_1)\cdots(z-\lambda_m),$$
    donde $\lambda_i$ son todos los eigenvalores de $T$.
\end{enumerate}

\subsubsection{Caracterización de polinomios que generan el operador cero}
Sea $T$ un operador lineal en $V$ y $q$ un polinomio sobre el
campo de $V$. Entonces $q(T)=0$ si y sólo si $q$
es un polinomio múltiplo del polinomio mínimo de $T$.

\subsubsection{Invertibilidad a partir del polinomio mínimo}
Sea $V$ de dimensión finita y $T$ un operador en este espacio.
Entonces $T$ no es invertible si y sólo si el término
constante del polinomio mínimo de $T$ es $0$.

\subsection{Eigenvalores en espacios reales de dimensión impar}
\subsubsection*{Kernel de dimensión par}
Si el campo son los reales y $V$ es de dimensión finita,
$T$ un operador en $V$, y además $b,c$ reales tal que 
$b^2<4c$. Entonces $\dim\ker (T^2+bT+cI)$ es par.

\subsubsection{Operadores de dimensión impar tienen eigenvalores}
Todo operador en un espacio vectorial de dimensión impar
tiene un eigenvalor.

\subsection{Matrices triangulares superiores}
\subsubsection{Diagonal de una matriz}
La diagonal de una matriz diagonal $A$ son sus elementos
$A_{i,i}$.

\subsubsection{Matriz triangular superior}
Una matriz es triangular superior si todas sus entradas
debajo de la diagonal son $0$. Esto es, si $A_{i,j}$
con $i>j$.

\subsubsection{Condiciones para una matriz triangular superior}
Suponga que $T$ es un operador lineal en $V$ y $v_1,\ldots,v_n$
una base.
Entonces los siguientes son equivalentes.
\begin{enumerate}
    \item $\M(T,v_i)$ es triangular superior.
    \item $\spn(v_1,\ldots,v_k)$ es invariante bajo $T$ para 
    cada $k=1,\ldots,n$.
    \item $Tv_k\in\spn(v_1,\ldots, v_k)$ para cada
    $k=1,\ldots,n$.
\end{enumerate}
\subsubsection*{Prueba}
Supongamos el punto 1. Como $T$
es un operador, sea $k\in\{1,\ldots,n\}$ entonces
$$Tv_k=\sum^n_{i=1} A_{i,k} v_i,$$
pero al ser $\M(T)$ triangular superior,
$$Tv_k=\sum^k_{i=1} A_{i,k} v_i.$$
Y si $v\in\spn(v_1,\ldots,v_k)$ es decir
$v=\sum^k_{i=1}\alpha_i v_i$, entonces
$$Tv=\sum^k_{i=1} A_{i,k} \alpha_i Tv_i,$$
$$Tv=\sum^k_{i=1} A_{i,k} \alpha_i \sum^k_{r=1} A_{r,k} v_r,$$
$$Tv=\sum^k_{r=1}\left(
    \sum^k_{i=1}A_{i,k}A_{r,k}\alpha_i\right) v_r.$$
De manera que $T(v)\in\spn(v_1,\ldots,v_k)$ probando que del
punto 1 sigue 2.

Es trivial que del punto 2 sigue el punto 3.

Supongamos el punto 3. De expandir $Tv_k$ para cada $k$
como combinación lineal de $v_1,\ldots,v_k$, se obtiene
$\M(T,v_i)$ por definición, y esta es
triangular superior.\hfill ////

\subsubsection{Una ecuación satisfecha por operadores con matriz triangular superior}
Sea $T$ un operador sobre $V$ que tiene una base, tal que
$\M(T)$ es triangular superior con entradas diagonales
$\lambda_1,\ldots,\lambda_n$. Entonces
$$(T-\lambda_1 I)\cdots(T-\lambda_n I) = 0.$$
\subsubsection*{Prueba}
Sea $v_1,\ldots,v_n$ una base para $V$ tal que sea $A=\M(T)$ sea 
diagonal.

Ya que $A-\lambda_k I$ es triangular superior con el
elemento $k$ de la diagonal igual a $0$, entonces
\begin{equation}
    \label{formasuma}
    (T-\lambda_k I)v_k=\sum^{k-1}_{i=1}A_{i,k} v_i.
\end{equation}

Cuando $k=1$ esto significa que
$$(T-\lambda_1 I)v_1 = 0.$$
Luego por conmutatividad de polinomios de operadores,
$$(T-\lambda_1 I)\cdots (T-\lambda_n I)v_1 = 0.$$

Ahora, suponiendo que para un $k$ tal que $1\leq k\leq n-1$
se cumple 
\begin{equation}\label{anuladores}
    (T-\lambda_1 I)\cdots(T-\lambda_m I)v_m = 0,
\end{equation}
para $m=1,\ldots,k$.

Por \ref{formasuma} sigue que
\begin{equation*}
    (T-\lambda_{k+1} I)v_{k+1} = \sum^{k}_{i=1}A_{i,k} v_i.
\end{equation*}
Le aplicamos $(T-\lambda_1 I)\cdots(T-\lambda_k I)$:
$$
(T-\lambda_1 I)\cdots(T-\lambda_k I)
(T-\lambda_{k+1} I)v_{k+1} =
\sum^{k}_{i=1}A_{i,k} (T-\lambda_1 I)\cdots(T-\lambda_k I)v_i.
$$
Por conmutatividad y \ref{anuladores}, podemos ordenar los 
operadores de los términos a la derecha,
tal que cada término $i$ se anula. De manera que
$$(T-\lambda_1 I)\cdots
(T-\lambda_{k+1} I)v_{k+1} = 0,$$
y por consecuencia también
$$(T-\lambda_1 I)\cdots(T-\lambda_{n} I)v_{k+1} = 0$$
por conmutatividad.

Por inducción hemos probado que
$$(T-\lambda_1 I)\cdots(T-\lambda_{n} I)v_k = 0$$
para cada $k$,
entonces el operador $(T-\lambda_1 I)\cdots(T-\lambda_k I)
(T-\lambda_{n} I)$ debe ser $0$.

\hfill ////

\subsubsection{Eigenvalores de una matriz triangular superior}
Sea $T$ un operador en $V$ con matriz triangular superior en
una base de $V$, entonces los eigenvalores de $T$ son
las entradas de la diagonal de esa matriz.
\subsubsection*{Prueba}
La demostración de Axler usa el teorema anterior.

\subsubsection{Equivalencia para una matriz trangular superior}
Sea $T$ un operador sobre un espacio vectorial $V$ finito.
entonces $T$ tiene una matriz triangular superior si y
sólo si el polinomio mínimo es
$(z-\lambda_1)\cdots(z-\lambda)$ para $\lambda_i$ escalares.

\subsubsection{Los operadores en campos complejos tienen matriz triangular superior}
Si $T$ es un operador en un espacio $V$ con campo complejo
entonces hay una base tal que $\M(T)$ es triangular superior.
\subsubsection*{Prueba}
Por el teorema anterior y la segunda versión del teorema
fundamental del álgebra.\hfill ////

\subsection{Operadores diagonalizables}
\subsubsection{Matriz diagonal}
Una matriz diagonal es una matriz cuadrada que tiene por
entradas $0$, excepto posiblemente en la diagonal.

\subsubsection{Operador diagonalizable}
Un operador $T$ sobre $V$ es diagonalizable si tiene
una base tal que $\M(T)$ es diagonal.

\subsubsection{Eigenespacio}
Si $T$ es un operador en $V$ y $\lambda$ un escalar de su campo.
El eigenespacio de $T$ correspondiente a $\lambda$ es
definido por 
$$E(\lambda, T)=\ker(T-\lambda I)=\{v:Tv=\lambda v\}.$$
Es decir, el espacio de todos los eigenvectores correspondientes
a $\lambda$ más el cero.

Con esta definición, $\lambda$ es un eigenvalor de $T$
si y sólo si $E(\lambda, T)\neq \{0\}$. Además es claro que es
un subespacio vectorial de $V$ pues el kernel de sus
mapeos lineales es un subespacio.

\subsubsection{La suma de eigenespacios es una suma directa}
Sea $T$ un operador en $V$ y $\lambda_1,\ldots,\lambda_m$ 
distintos eigenvalores de de $T$. Entonces
$$E(\lambda_1, T)+\cdots+E(\lambda_m, T)$$
es una suma directa.
Además si $V$ es de dimensión finita, entonces
$$\dim E(\lambda_1, T)+\dim E(\lambda_m, T) \leq \dim V.$$

\subsubsection{Condiciones equivalentes para diagonalizar}
Sea $T$ un operador lineal sobre $V$, y
$\lambda_1,\ldots,\lambda_m$ los distintos eigenvalores de
$T$. Entonces los siguientes son equivalentes.
\begin{enumerate}
    \item $T$ es diagonalizable.
    \item $V$ tiene una base que consiste de eigenvectores de
    $T$.
    \item $V=E(\lambda_1, T)\oplus E(\lambda_m, T)$.
    \item $\dim V =
    \dim E(\lambda_1, T)+\ldots+\dim E(\lambda_m, T)$.
\end{enumerate}

\subsubsection{Suficientes eigenvalores implcian diagonalizabilidad}
Si $T$ es un operador lineal sobre un espacio $V$ de dimensión
finita y $T$ tiene $\dim V$ eigenvalores, entonces
$T$ es diagonalizable.

\subsection{Operadores conmutativos}
\subsubsection{Operadores conmutativos corresponden a matrices conmutativas}
Suponga que $S$ y $T$ son operadores en $V$ de dimensión finita
y $v_i$ una base. Entonces $S$ y $T$ conmutan si y sólo si
$\M(T)$ y $\M(S)$ conmutan.

\subsubsection{Eigenespacio es invariante sobre operador
conmutante}
Si $S,T$ son operadores conmutantes en $V$, y $\lambda$ un
escalar. Entonces $E(\lambda, S)$ es invariante bajo $T$.

\subsubsection{Diagonalizabilidad simultanea equivale a conmutatividad}
Dos operadores diagonalizables tienen matrices diagonales
en la misma base si y sólo si los operadores son conmutativos.

\subsubsection{Eigenvector común para operadores conmutativos}
Todo par de operadores conmutativos en un espacio vectorial
complejo finito distinto de $\{0\}$ tiene un
eigenvector en común.

\subsubsection{Operadores conmutativos tienen simultaneamente
matrices triangulares superiores}
Si $S,T$ son operadores conmutativos en $V$, entonces
hay una base de $V$ respecto a la cual ambos operadores
tienen matrices triangulares superiores.

\subsubsection{Eigenvalores de suma y producto de operadores conmutativos}
Sea $V$ un espacio vectorial finito complejo, y $T, S$
operadores conmutativos en este mismo. Entonces
\begin{itemize}
    \item todo eigenvalor de $S+T$ es la suma de un eigenvalor de $S$ y uno de $T$.
    \item Todo eigenvalor de $ST$ es el producto de un eigenvalor de $S$ con un eigenvalor de $T$.
\end{itemize}

\newpage
\section*{Referencias}
\begin{enumerate}
    \item Linear Algebra Done Right 4ta edición, por Axler, Sheldon.
\end{enumerate}
\end{document}
